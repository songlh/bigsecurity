\vspace{-0.05in}
\section{Introduction}
\label{sec:intro}

Malware grows exponentially~\cite{avtest} and places an imperative threat to human society. 
For example, more than 390000 new malware are registered in AVTest institute every day~\cite{avtest}.
As another example, a new type of threat, ransomware, cost more than 1 billion dollars in 2016~\cite{ransomware}. 
%To fight against malware and detect these new types of malware, 
%anti-virus tools are improving rapidly,
%by constantly updating their signature databases, 
%by using more advanced techniques like deep learning~\cite{cylance}, 
%or by utilizing more data. 

To fight against malware, it is essential to understand existing anti-virus engines.
There are many anti-virus engines in the real world~\cite{virustotal}.
One question that has never been studied before is: 
{\em is there influence across different anti-virus vendors?}
That is, can an anti-virus vendor's decision is affected 
by other vendors' malware detection results. 
%The answer to this question can help individual users better select anti-virus engines,
%and help the whole security community better aggregate detection results from different vendors. 

Fortunately, we live in the ``big data'' era and there is no shortage of data on malware 
and different anti-virus engines' 
detection results. 
One such type of data repositories is online malware analysis services. 
There are many online malware analysis services~\cite{virustotal,malwr,vxstream,anubis} 
that use malware sandboxes and state-of-the-art
anti-virus engines to analyze user-submitted files 
or URLs and produce detailed detection reports. 

We propose to use the vast amount of malware big data in online malware services 
to study the relationship between different anti-virus engines, 
to quantify the influence across them, 
and to predict whether one engine's detection results will be affected by other vendors. 
To achieve all these goals, we utilized the biggest open data repository
that contains billions of real-world malware, {\em \vt}.

\vt{} is a free online malware scanning service
that applies a set of state-of-the-art anti-virus engines to analyze user-submitted files 
and sends a detection report back to user.
\vt{} provides public access to all its submitted files and analysis results. 
\vt{} is a valuable resource to study and 
understand real-world anti-virus engines for the following two reasons. 

First, there are a huge amount of suspicious files submitted to VirusTotal. 
For example, within our data collection window, 
there were around 40 million submissions to \vt\ each month. 
These submissions cover a large variety of file types and 
are conducted by a large variety of \vt\ users from all over the world. 
This amount of diverse data on VirusTotal serves as a 
good representation of malware in the real world to study influence across anti-virus vendors. 

Second, for almost all submissions, 
VirusTotal applies no less than 50 state-of-the-art anti-virus engines to analyze them. 
These engines cover almost all popular anti-virus engines.
VirusTotal keeps detailed engines' detection results and provides an open access to these results. 
Analyzing historical detection results can help capture how influence propagates among different anti-virus vendors. 

In this paper, we conduct an extensive, 
large-scale study on anti-virus engines
using 4 months' data collected from \vt.
Specifically, we study the question of whether or not different anti-virus vendors can influence each other.
Anecdotally, anti-virus vendors frequently leverage VirusTotal to identify false negatives in their products, 
which are malware detected by others' products but not detected by their own products. 

To verify this hypothesis, we used the detection history from VirusTotal to create a new type of graph that we call 
{\em influence graph} to model the influence relationship across vendors.
We further developed a set of statistical models on top of the influence graph to quantify the influence between vendors
and a prediction model to predict how likely the decision of a vendor is influenced by others. 
With this method, we confirmed that there does exist high influence between vendors;
certain vendors are highly influenced by the detection results of other vendors 
and use this information to change their detection results.

Our study advances the understanding of anti-virus engines in the real world 
and provides various valuable insights for future researchers and vendors, 
some of which are surprising and have never been revealed in the past.
Specifically, this paper makes the following main contributions.
%We hope that our findings and our methodologies can help future security researchers and practitioners 
%to better understand anti-virus engines and better leverage their detection results.


%\vspace{-0.05in}
\begin{itemize}
%\item 
%Online malware detection services offer rich sets of real-world data that are representative of 
%both the latest malware and malware in history.
%Our study show that it is not only feasible and but also valuable to study these data in large scale.
%Thus, we call attention from our research community to investigate these online repositories more closely.

\item
As far as we know, we are the first to perform a big data analysis on anti-virus engines themselves.
Our influence study results in the surprising finding that there are anti-virus vendors who are influenced by almost all other vendors,
while some vendors influence many other vendors. 
This result alerts normal users and security experts to be treat detection results from anti-virus vendors with more caution.

\item
We developed a set of analytical methodologies, statistical models, and prediction methods to study anti-virus engines.
They can assist future researchers and practitioners to perform more analysis 
on other malware data and develop more advanced analytical techniques.


\end{itemize}
