%Title: 
%1. Learning from VirusTotal
%2. A Glimpse of VirusTotal
\vspace{-0.1in}
\section{Introduction}

Malwares grow exponentially~\cite{avtest}. 
AVTest~\cite{avtest} reports that more than 140 million new malwares appeared in 2015. 
Such malwares are posing increasing threats to the human society every day. 
For example, there were almost 2 million attempts to 
steal money from online bank accounts 
with malwares that exploit vulnerabilities in the Adobe Flash player~\cite{kaspersky}. 

Researchers and practitioners continue to build security tools to defend against new malwares.
To assist the design of these tools, it is essential to understand malwares in the real world. 

Previous works on analyzing the behaviors and evolutions of malwares~\cite{ZhouSP2012,GuptaComsnets2009} 
have provided insights 
into how malwares circumvent the detection from existing antivirus techniques and how malware writers create new malwares. 
However, these works only studied a limited amount of malwares that are targeted for certain types of security threats or antivirus engines.

Studying malwares in a large scale and with high diversity, what we call {\em big malwares}, 
can expose new insights beyond these isolated studies.

%VirusTotal~\cite{virustotal} is a popular online service that real-world users use to analyze suspicious files and URLs.
%It applies more than 50 antivirus engines to each submitted file 
%to detect various kinds of malwares including viruses, worms, and trojans. 
%It then generates a summary report that includes the detection results of all these engines. 
%VirusTotal saves and provides an open access to all user-submitted files and generated reports. 


VirusTotal~\cite{virustotal} is a popular online service that real-world users use to analyze suspicious files and URLs.
{\color{red} (Writing 2 and 5) VirusTotal does not judge whether a submission is malware by itself. Instead,
it applies state-of-the-art antivirus engines from more than 50 vendors to each submitted file, 
and generates a summary report that includes the detection results of all these engines.} 
VirusTotal saves and provides an open access to all user-submitted files and generated reports. 


The VirusTotal repository provides a valuable resource to gain insights into 
the behavior of malwares.
%Inspired by previous work on mining software repositories~\cite{GuoICSE2010,bigcode, big-lessons,big-translation,code-completion,big-predicting} 
%in the software engineering and programming languages community, 
%we believe that leveraging data on VirusTotal could also enable many ``big security'' applications.   
%
First, it contains a huge amount of real-world files.
For example, there were more than 40 million suspicious files submitted in November 2015 (Figure~\ref{fig:subnum}). 
Files from VirusTotal were submitted by real-world users from all over the world since 2004
and involve various security threats. 
This amount of diverse data makes VirusTotal a good representative of malwares in the real world. 

Second, VirusTotal applies a host of state-of-the-art antivirus engines to all submitted files,
VirusTotal captures how these engines evolve over time.

Third, VirusTotal provides rich metadata. 
%VirusTotal updates each antivirus engine every five minutes. 
Besides reporting whether or not a submitted file has malware, 
VirusTotal also captures the exact type of malware, 
which antivirus engines detected the malware,
when and by whom the file is submitted,
and other useful metadata.
There are also active malware researchers and engineers
who comment and vote on each submitted file, 
providing valuable human inputs. 

The VirusTotal repository exposes many new research opportunities.
%List all the possible research questions that one can answer by analyzing VT data.
%For example, ...
For example, studying malwares over a long time period and across all countries in the world 
can provides a high-level insight into how malwares evolve over time and over geo-locations.
Studying how antivirus engines change over time together with how malwares change can 
reveal the effectiveness of responses to new security threats.
Finally, it is interesting to investigate if we can build online malware prediction tools 
by applying machine learning techniques on the VirusTotal data.
All these insights could in turn assist future antivirus researchers and engineers
to better design malware defense mechanisms.
%we can figure out what vulnerabilities malware writers prefer to exploit, 
%and design new systems and programming languages without those vulnerabilities. 
%For example, most malwares are created automatically. 
%Could we catch malwares' evolution pattern, and automate the updates of antivirus techniques?
%For example, there are increasing interests in applying machine learning techniques in different areas. 
%Could we leverage VirusTotal data to train a machine learning model and detect malwares we have not seen before? 

\begin{figure*}[!htb]
\minipage{0.31\textwidth}
  \includegraphics[width=\linewidth]{figure/type}
  \caption{File type distributions.
(File types and their distributions for all VirusTotal submissions from 05/07/2016 to 09/06/2016.)
}
\label{fig:type}
  %\label{fig:overlap}
\endminipage\hfill
\minipage{0.31\textwidth}
  \includegraphics[width=\linewidth]{figure/Submissions}
 \caption{The number of files and PE files.
%{\footnotesize{
(The number of suspicious files and the number of PE files submitted to VirusTotal from 05/07/2016 to 09/06/2016.)
%}
}
\label{fig:subnum}
  %\label{fig:maxUncover}
\endminipage\hfill
\minipage{0.31\textwidth}%
  \includegraphics[width=\linewidth]{figure/countryPie}
\caption{How PE submissions distribute among different countries.
%\footnotesize{
(Only countries with more than 1\% PE submissions are shown.)
%}
}
\label{fig:countryPie}
\endminipage\hfill

%\vspace{-0.2in}
\end{figure*}

Unfortunately, there has been little work in looking at this valuable repository.
In industry, many antivirus vendors use VirusTotal to identify false negatives 
and false positives in their products. 
However, they only use VirusTotal to examine suspicious files separately, 
and do not consider correlations among different suspicious files. 
In academia, only until recent did researchers begin to pay attention to mining the VirusTotal repository. 
Graziano \etal~\cite{neeles} leveraged VirusTotal user ID information to identify malware writers 
who use VirusTotal as a test platform. 



We propose to investigate in the VirusTotal repository along two directions:
offline study of its rich data and metadata, 
and online analysis and prediction of malwares.
%The offline study can be conducted along several lines: 
%downloading malware samples and analyzing them to understand their static features and how malwares evolve; 
%studying VirusTotal metadata %including submission information and reports from different vendors, 
%to understand general characteristics of malwares in the real world and correlations among different malwares; 
%and applying machine learning techniques on VirusTotal data to build malware prediction tools.
%leveraging behavior information on virustotal, and run malwares in virtual machines to understand their dynamic behaviors. 

As a first step, we collected one month of VirusTotal repository data with more than 40 million suspicious files
and conducted an early-stage empirical study on them. 
{\color{red} (Writing 10) VirusTotal supports downloading both data files and metadata of them.
In this study, we focus on studying metadata and found that just by analyzing metadata, 
we can already find a fair amount of valuable conclusions.
Studying data could potentially provide more insights and we leave it for future work.}
%There is a tradeoff between analyzing accuracy and effort. 
%Analyzing metadata is just a starting point, and we leave other information on VirusTotal in the future. }
%An empirical study is the prerequisite to conduct data mining on VirusTotal. 
%We first collect more than 40 million suspicious file submissions from VirusTotal.
% this para needs total rewriting
%Using antivirus engines and malwares under the same platform improves malware detection %accuracy. 

We centered our analysis around the concept of {\em malware families}.
A malware family is a set of malwares that have common behaviors.
We focus our initial attention on all Windows executable (PE) malwares detected by Microsoft
antivirus engines, 
because our previous experience shows that Microsoft engines can generate precise results on PE files.
%Specifically, we analyzed malware characteristics including
%the submission frequencies and generation rates of malware families, 
%the temporal properties of malwares, 
%and the distribution of malwares across malware families.

We focus our analysis on two dimensions: temporal properties of malwares and 
malware family distribution.
In each dimension, we investigate both offline and online analysis mechanisms.

We began our temporal analysis with general malware characteristics including the submission
frequencies and generation rates of malware families. 
We then study the burstiness of malwares---how closely malwares belonging to the same family occur in time. 
To understand malware burstiness, we designed a new cache-based mechanism 
that can support both offline and online analysis.
%Our LRU malware family cache can achieve a 90\% cache hit rate by only using 100 cache entries. This experimental result verifies the burstiness of malwares. 
Moreover, our LRU cache can predict malware families occurrences in the near future, 
allowing antivirus vendors to better focus their efforts. 

We focus our malware family distribution study on detecting skewness of malware families and identifying hot families---families that have many malwares.
We observe that the distribution of malware families is highly skewed. 
To assist online analysis of malware family distribution, 
we apply a frequent item mining algorithm~\cite{space-saving} to identify hot malware family. 
This solution can answer historical hot malware families in nearly-real time using constant memory. 


%%% the following para needs total rewrite
%We further study the burstiness of malwares---how closely malwares belonging to the same family appear in time. %%% is this def correct?
%%% I don't understand the following sentence
%We built a cache-based malware prediction tool that predicts malwares in which families would appear in the near future. 
%Our malware family cache can achieve a 90\% cache hit rate by only using 100 cache entries.
%Our technique would allow antivirus vendors to focus their efforts. 
%We also observe that distributions of malware families are highly skewed. 
%This observation inspires us to apply a frequent item mining algorithm to 
%identify hot malware families. 
%We view malware submissions as a stream based on their submission timestamps and 
%feed this stream into space saving algorithm~\cite{space-saving}. 
%Our solution can precisely answer hot malware family queries in nearly-real time, by using a constant number of counters.  

%\begin{figure}[t!]
\begin{center}
\includegraphics[width=2.5in]{figure/nov}
\mycaption{fig:subnum}{The number of files and malwares.}
{
The number of suspicious files and the number of malwares submitted to VirusTotal in November 2015. 
}
\end{center}
\vspace{-0.5in}
\end{figure}





In summary, this paper makes the following contributions:

\begin{itemize}

\item We are the first to collect mass data from the VirusTotal repository for large-scale malware study.
We collected and preprocessed files submitted to VirusTotal in November 2015 (Section~\ref{sec:meth}).
%and analyzed their general characteristics. 
%We find that most malware files are only submitted once to VirusTotal (Section~\ref{sec:meth}) 
%and that roughly 100-400 new malware families appear each day (Section~\ref{sec:temporal}). 

\item We studied the burstiness of malwares using a new 
cache-based malware prediction tool.
This tool achieves greater than 90\% prediction precision (Section~\ref{sec:temporal}). 

\item We observe that family distributions of malwares are highly skewed, 
\ie, some family has a huge amount of malwares while others only include a few malwares. 
Leveraging this observation, we built a hot malware family mining tool that identifies hot 
malware families (Section~\ref{sec:dist}).
%%% I don't understand this: using a constant number of counters

\item We identify several key research opportunities on top of the VirusTotal repository (Section~\ref{sec:oppo}). 

\end{itemize}


