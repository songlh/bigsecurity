\section{Related works}

Many research efforts~\cite{bigcode, big-lessons,big-translation,code-completion,big-predicting} 
have been spent to explore how to leverage ``big code'' repository, 
such as GitHub, BitBucket, and CodePlex, and these techniques inspire us to explore how to leverage data on VirusTotal. 
SLANG~\cite{code-completion} can fill uncompleted programs with call innovations, 
by using statistical models trained from extract sequences of API calls from large code bases.  
JSNICE~\cite{big-predicting} could predict identifiers' types and obfuscated identifiers' names for Javascript programs. 
JSNICE translates programs into dependence graphs and learns a CRF model by using a large training set. 
All predictions are made by optimizing a score function based on the learned CRF model. 
\citet{big-translation} apply phrase-based statistical translation approaches to translate C\# program to Java.
To sum up, all these techniques are built based on source code repositories, 
and their goals are to improve the development stage. 
However, VirusTotal is a repository containing binary malwares, 
and the goal to conduct data mining on VirusTotal data is to improve antivirus techniques. 

There are existing works~\cite{hacker-vt,neeles} to conduct data mining on VirusTotal data to identify malware development cases, 
where malware writers use VirusTotal as a testing platform and 
try to develop malwares which can not be detected by antivirus engines. 
These techniques utilize submission\_id information and similarity between malwares, which are different from information we use. 
We believe other information on VirusTotal could also be leveraged in the future. 

\section{Conclusion}
