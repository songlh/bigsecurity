\section{Introduction}
In current cloud-based anti-malware services~\cite{xxx}, the lightweight agent on a protected endpoint offloads the majority of data analysis to the provider's infrastructure. This approach has growing popularity among anti-malware service providers and customers. As forecasted, the market of mobile cloud-based anti-malware is expected to have a growth of 17\% annually in the next 5 years~\cite{report2016}.

Despite the growing applicability of malware repository in clouds, there are very few systematic measurement studies to guide the practical research and engineering. Crucially, little is known about the temporal characteristics and distribution of malware families\footnote{Categories output by antivirus engine, defined in Section XX}, which should have revealed valuable research and commercial opportunities.

In this paper, we collect a malware dataset from VirusTotal and conduct an empirical study on it. And then we focus on the temporal characteristics and distribution of malware families. In dataset collection, We first collect more than 40 million suspicious file submissions on VirusTotal, and then remove redundancy and narrow down the scope to Windows executables and libraries to guarantee analytical accuracy.

In the temporal analysis, we find that the new malware families appear at a rate of 100-400 each day. In addition, new malware families appear in bursts. We build a cache-based malware prediction tool to validate this observation. The cache-based malware prediction tool aims to predict which malware family would appear in the near future, and the cache achieves 90\% hit rates by only using 100 cache entries.

In the study of malware families, we observe that the distribution of malware families is highly skewed, which implies that antivirus vendors can focus their effort if they can identify hot malware families.
Thus, we further improve hot malware identification methodology using frequent item mining algorithm~\ref{xxx}.
Our experiments show that this approach can precisely answer hot malware family query in nearly-real time, by using a constant number of counters. 

To sum up, we made the following contributions in this paper:
\begin{itemize}
\item We collect a malware dataset from VirusTotal, which has over 40 million records. We overcome the difficulty of eliminating redundancy and filtering data to guarantee analytical accuracy (Section~\ref{XXX}).

\item We summarize the new malware family appearing rate. And we observe the bursty appearance of new malware family, and validate this observation using a cache-based malware prediction tool (Section xx).
\item We observe that the distribution of malware families is highly skewed. Leveraging this observation, we build a hot malware family mining solution, which can identify hot malware families by using a constant number of counters (Section~\ref{sec:dist}).
\item We discuss the future research opportunities through our mining on VirusTotal data(Section~\ref{sec:oppo}). 
\end{itemize}
