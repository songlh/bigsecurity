\section{Influence model}

%\subsection{Influence graph}

\textcolor{gray}{How we model vendors? influence}

As we have mentioned in section (blah), vendors could rely on each other to decide their own detection results. 
To measure such influence among vendors, we adapt the static influence model on social networks by Goyal et al. \cite{goyal2010learning}. 
This model takes the a list of actions as input, and calculates a graph of influence among nodes in a social network. 
We assume each vendor could influence any other vendors, and calculate the influence on a complete graph. 
Since we use a complete graph in this paper, the model is slightly different from the original model. 
We introduce our influence model in this section.

%\subsection{Basic Definitions}
First, we introduce some basic definitions of action, and how to generate an action $(v, a , t)$ or actions in action list $Actions$ from a time sequence $s$. 
For a vendor $v$, it could mark a file $f$ as malicious or benign at some day $d$. 
We denote the action `marking as malicious' as $(v, f_1, d)$ and `marking as benign' as $(v, f_0, d)$.  
\textcolor{red}{Currently, we consider all the detection results in each day as an action. Should we eliminate some of the actions?}

Then we define action propagation among vendors. 
We denote $prop(a, v_i, v_j, \Delta d)$ as ``an action $a$ propagates from $v_i$ to $v_j$ in $\Delta d$ days'', where $\Delta d = d_j-d_i$. 
This happens iff. $\exists (v_i, a, d_i), (v_j, a, d_j) \in Actions$ with $d_j-d_i>d_w$, where $d_w$ is the time window of vendors reference other vendors. \textcolor{red}{We would like to know the results under different time windows.} 

Now we define how to calculate the influence among vendors. 
We write $A_v$ as the number of actions performed by vendor $v$. 
Formally, we have 
\begin{equation}
A_v = |\{(u, a, d) | (u, a, d) \in Actions \land u=v\}| 
\end{equation}

Then, we consider the actions that two vendors $u$ and $v$ performed together: 
\begin{multline}
A_{u\&v} = |\{(u, a, d) | (u, a_u, d_u) \in Actions \\ \land (v, a_v, d_v) \in Actions \land a_u=a_v \}|
\end{multline}

The number of actions that performed by either vendors could be calculated as: 
\begin{equation}
A_{u|v} = A_u+A_v-A_{a\&v}
\end{equation}

We also have 
\begin{multline}
A_{u2v} = |\{(u, a_u, d_u) | (u, a_u, d_u) \in Actions \\ \land (v, a_v, d_v) \in Actions \land a_u=a_v \land d_u<d_v \}|
\end{multline}
 as the number of actions propagated from $u$ to $v$. 

With the values, we could calculate the influences, which is defined on a complete graph $G= (V,E)$. $V$ is the set of vendors, and $\forall u,v \in V, \exists (u,v,p_{u,v}) \in E$, where 
$p_{u,v}$ is the influence probability of $v$ influenced by $u$. 

\subsection{Influence measurement}
We have 4 metrics of calculating probability $p_{u,v}$ as follows:

\textbf{Bernoulli Distribution:} This model estimates $p_{u,v}$ as the ratio of the number of actions propagated from $u$ to $v$ over the total number of actions taken by $u$.

\begin{equation}
p_{u,v} = \frac{A_{u2v}}{A_u}
\end{equation}

\textbf{Jaccard Index:} This model estimates $p_{u,v}$ as the ratio of the number of actions propagated from $u$ to $v$ over the total number of actions taken by either $u$ or $v$.

\begin{equation}
p_{u,v} = \frac{A_{u2v}}{A_{u|v}}
\end{equation}

We also consider partial credit to improve the two metrics above.
The partial credit is explained as follows. When $v$ takes an action $a$, it may be influenced by the combination of all its neighbors taking the action $a$ before $v$. To account for this effect, we use partial credit and calculate the partial credit for $u$ who takes an action $a$ before $v$ as

\begin{equation}
credit_{u,v}(a) = \frac{1}{\sum_{w \in V}I(\exists a. prop(a, w, v, \Delta d))}
\end{equation}

where $I$ is the indicator: $I(P)$ equals to 1 when $P$ is true, otherwise it is 0. Then we have another two metrics:

\textbf{Bernoulli Distribution with Partial Credit:} This model estimates $p_{u,v}$ as the sum of all partial credits taking by u for actions propagated from $u$ to $v$, dividing by the number of actions taken by $u$.

\begin{equation}
p_{u,v} = \frac{\sum_{a\in A_{u2v}}credit_{u,v}(a)}{A_{u}}
\end{equation}

\textbf{Jaccard Index with Partial Credit:} This model estimates $p_{u,v}$ as the sum of all partial credits taking by u for actions propagated from $u$ to $v$, dividing by the number of actions taken by $u$.

\begin{equation}
p_{u,v} = \frac{\sum_{a\in A_{u2v}}credit_{u,v}(a)}{A_{u|v}}
\end{equation}

%How we generate the dataset is illustrated as in Figure \ref{fig:action}. In this figure, there are three vendors, $u$, $v$, and $w$. $v$ flips at time 4 and 9, from benign to malicious. $w$ flips from benign and malicious at time 5. For $v$, it could be influenced by $u$ at time 3 and time 8, and $w$ at time 8. For $w$, it could be influenced by both $u$ and $v$ at time 8. Therefore, for this small case, we have the following action propagations: $(u, f_1, 3) \to (v, f_1, 4)$, $(u, f_1, 8) \to (v, f_1, 9)$, and $(w, f_1, 8) \to (v, f_1, 9)$.

\subsection{Results}

a. On the small data set

b. On the large data set**

Justify file category

Justify length of the data

\subsection{Discussion}
