\section{Introduction}
\label{sec:intro}

%There is and will continue to be a constant competition between anti-virus tools and malware.
Malware grow exponentially~\cite{avtest} and place an imperative threat to human society. 
For example, more than 390000 new malware are registered in AVTest institute every day~\cite{avtest}.
As another example, a new type of threat, ransomware, has caused more than 1 billion dollars this year~\cite{ransomware}. 
To fight against malware and detect these new types of malware, anti-virus tools are also improving rapidly,
by constantly updating their signature database, by using more advanced techniques like deep learning~\cite{cylance}, or by utilizing more data. 

In order to improve anti-virus tools and defend against emerging threats from malware, 
it is essential to understand malware and existing anti-virus tools in the real world. 
Previous works on studying malware and anti-virus engines do provide valuable 
insights~\cite{ZhouSP2012,GuptaComsnets2009, vendors-study} such as  
how malware writers create new malware and how malware escape from the detection of anti-virus engines.
However, most previous works focus only on one or few problems of malware with limited scale. 
There is also a lack of study on anti-virus engines
and how they relate to malware.  

Fortunately, we live in the ``big data'' era and there is no shortage of data on malware and anti-virus
engines in the real world. One such type of data repositories is online malware analysis services. There
are many online malware analysis services~\cite{virustotal,malwr,vxstream,anubis} that use malware sandboxes and state-of-the-art
anti-virus engines to analyze user-submitted files or URLs and produce detailed detection reports. 

We propose to use the vast amount of malware big data in online malware services
to study real-world malware and their relationship with anti-virus engines.
To conduct such study, we utilized the biggest open data repository
that contains billions of real-world malware, {\em \vt}.

\vt{} is a free online malware scanning service
that applies a set of state-of-the-art anti-virus engines to analyze user-submitted files 
and sends a detection report back to user.
\vt{} provides public access to all its submitted files and analysis results. 
\vt{} is a valuable resource to study and 
understand real-world malware and anti-virus engines for the following reasons. 

First, there are a huge amount of suspicious files submitted to VirusTotal. 
%As shown in Figure~\ref{fig:subnum}, 
For example, within our data collection window, 
there were around 40 million submissions to \vt\ each month. 
These submissions cover a large variety of file types and 
are conducted by a large variety of \vt\ users from all over the world. 
This amount of diverse data on VirusTotal serves as a good representation of malware in the real world.  

Second, for almost all submissions, 
VirusTotal applies no less than 50 state-of-the-art anti-virus engines to analyze them. 
VirusTotal keeps detailed detection results and provides an open access to these results. 
Analyzing historical detection results can help capture how anti-virus engines evolve over time. 

Third, VirusTotal provides rich metadata for each submission. 
Besides detailed detection results from various anti-virus engines, 
VirusTotal also provides file type information, submitter information, and a hash string of the original submitted file.
These {\em metadata} can be used to better understand malware 
and assist both security experts and normal users in malware detection.
%which can help categorize malware, 
%source ID (country), which can help understand popularity of malware, 
%ssdeep digest string, by using which we can calculate code similarity without accessing binary executable, and so on. 

Unfortunately, there has only been limited attention to \vt\ in the past. 
%Researchers have used \vt\ as a testing platform when \yiying{developing malware?}
Researchers tried to capture malware writers who leverage \vt{} as the testing platform during malware development~\cite{huangvt2016bigdata, neeles}. 
Anti-virus vendors use \vt\ to detect possible false positives and false negatives in their products. 
But none of them study the rich malware data provided by \vt\ in a large scale. 

In this paper, we conduct an extensive, large-scale study on both malware and anti-virus engines
using data collected from \vt.
We collected 4 months of metadata of all file submissions to VirusTotal, 
a total of 151 million submissions and 120 million files.
We focus on analyzing malware files and leave URLs for future work, 
since file submissions contribute to the majority of \vt\ repository.
%Following previous works~\cite{SongAPsys2016} on studying VirusTotal,
%We focus our effort on Windows \textit{Portable Executable} ({\em \pe}) files, 
%which account for more than half of \vt{}’s submissions.

After collecting and pre-processing data from \vt, 
we first perform a basic analysis to gain an overall knowledge of the \vt\ data repository.
This analysis answers a rich set of fundamental questions about the \vt\ data repository
including what types of files and how many of them were submitted to \vt;
who submitted files to \vt, at what time, and from where;
how big are the files submitted to \vt;
how many anti-virus engines were used to analyze a submission and how many of them labeled it as malware.
%Our answers to these questions 

On top of these basic findings, we developed a set of more advanced studies in two directions. 
First, we study the correlation between submissions' metadata and their {\em detection rates}, 
the percentage of engines labeling a submission as malware. 
We found high correlation between detection rate 
and three factors: submission file size, the history of submitting a file, and the reputation of the submitters.
We further developed a regression model to capture the effect of all these factors on detection rates
and found that using just these three factors, we can predict malware with a much higher accuracy than random guess.
These results shed light on what types of submission are more likely to be malware.
With this result, future researchers and anti-virus vendors can have more guided direction 
into what they should have further investigation.
%help security experts invest their limited manual efforts, 
%and help anti-virus vendors identify possible false positives and false negatives in their products.   

To fight against malware, it is not enough to just understand malware;
we should also understand anti-virus engines and their detection results.
Specifically, we study the question of whether or not different anti-virus vendors can influence each other
and if detection rate is a perfect measurement of the likelihood of a file being a true malware.
Anecdotally, anti-virus vendors frequently leverage VirusTotal to identify false negatives in their products, 
which are malware detected by others' products but not detected by their own products. 

To verify this hypothesis, we used the detection history from VirusTotal to create a new type of graph that we call 
{\em influence graph} to model the influence relationship across vendors.
We further developed a set of statistical models on top of the influence graph to quantify the influence between vendors
and a prediction model to predict how likely the decision of a vendor is influenced by others. 
With this method, we confirmed that there do exist high influence between vendors;
certain vendors are highly influenced by the detection results of other vendors 
and use this information to change their detection results.
%Anti-virus vendors can rely on this technique to identify false negatives in their products. 

Our study advances the understanding of malware and anti-virus engines in the real world 
and provides various valuable insights for future researchers and vendors, 
some of which are surprising and have never been revealed in the past.
Specifically, this paper makes the following main contributions.
We hope that our findings and our methodologies can help future security researchers and practitioners 
to better understand malware and fight against them.

\vspace{-0.05in}
\begin{itemize}
\item 
Online malware detection services offer rich sets of real-world data that are representative of 
both the latest malware and malware in history.
Our study show that it is not only feasible and but also valuable to study these data in large scale.
Thus, we call attention from our research community to investigate these online repositories more closely.
%\vt\ constantly receive a huge amount suspicious samples from all over the world. 
%Malware on \vt\ provides a representative sample of malware in the real world.
%Research communities should pay more attention to this valuable data source. 


\begin{table}[h!]
\centering
\footnotesize
{
\begin{tabular}{l|l}
\hline
Metadata Field & Explanation \\
\hline                            

{\bf size} & file size \\
{\bf type} & file type \\
{\bf tags} & labels with more specific information for each {\bf type}\\
{\bf first\_seen} & when the file was first submitted \\
{\bf last\_seen} & when the file was last submitted \\
{\bf hashes or digests} & sha1, sha256, md5, vhash, and ssdeep\\ \hline

{\bf scan\_date} & the latest scanning date for the file  \\
{\bf scan\_id} & ID of the scanning record \\
{\bf verbose\_msg} & a string for detailed scanning status \\
{\bf total} & \# of engines analyzing the file \\
{\bf positives} & \# of engines that flagged the file as malicious \\
{\bf scans} & a map containig engines' detailed analysis results \\
	\qquad {{\bf detected}} & true if the engine's label is malicious \\
	\qquad {{\bf version}}  & the version of the engine \\
	\qquad {{\bf result}}   & malware label \\
	\qquad {{\bf update}}   & the update date of the engine \\
\hline
\end{tabular}
\vspace{0.1in}
}
\mycaption{tab:fields}{VirusTotal Metadata.} 
{Fields retrieved from VirusTotal \texttt{report} and their explanation.
Some fields unrelated to this paper are not included.}
%\label{tab:fields}
\end{table}








\item
We show that with just file metadata, we can already achieve malware detection that is significantly more accurate than random guess.
%Several metadata properties are highly correlated with detection rate. 
%Our regression model based on these properties can effectively predict detection rate. 
Researchers and anti-virus vendors can leverage these properties 
to perform initial filtering through the vast amount of file submissions
and focus their efforts on more suspicious files.
Moreover, although still not comparable to anti-virus engines that scan files to detect malware, 
our results open up the possibility of using metadata only to detect malware. 
The implication is huge: users may not need to transfer their original files to an untrusted anti-virus service
but only submit file metadata.

\item
As far as we know, we are the first to perform a big data analysis on anti-virus engines themselves.
Our influence study results in the surprising finding that there are anti-virus vendors who are influenced by almost all other vendors,
while some vendors influence many other vendors. 
This result alerts normal users and security experts to be treat detection results from anti-virus vendors with more caution.

\item
We developed a set of analytical methodologies, statistical models, and prediction methods to study malware and anti-virus engines.
They can assist future researchers and practitioners to perform more analysis on other malware data and develop more advanced analytical techniques.
%For example, we developed a new way to analyze the anti-virus engine influence problem based on influence graph;
%such type of graph-based analysis is first used in analyzing 
%Our influence model borrowed from social media analysis can help predict anti-virus vendors' detection behaviors, 
%which indicate the feasibility to apply social media models to security area. 
%More future work should be spent here. 

\end{itemize}

The rest of this paper is organized as follows.
Section~\ref{sec:meth} introduces \vt\ and discusses how we collect data from \vt. 
Section~\ref{sec:basic} presents the basic analysis of the \vt\ data we collected. 
Section~\ref{sec:corr} studies the correlation between various metadata properties and detection rate. 
Section~\ref{sec:influ} analyzes how anti-virus vendors influence each other.
Finally, we discuss related works in Section~\ref{sec:related} and conclude in Section~\ref{sec:con}.