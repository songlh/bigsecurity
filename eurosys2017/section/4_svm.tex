\section{Malware Classification}
\label{sec:ssdeep}

Machine learning systems are just as good
as the data they consumed---One potential 
use case of the massive amount of data 
available on VirusTotal is to build 
machine learning models for applications
such as malware classifications and
clustering. In this section, we study
the question that {\em Is the data
provided by VirusTotal enough to support
classifications and clusterings tasks?}

Our study reveals both positive and
negative answers to this question---For
some classification tasks, we are able
to build an automatic classifier whose
accuracy is higher than 80\% by just
using the static signature provided by VirusTotal
and we expect the quality to keep increasing
given more data; on the other hand,
we identify some classification tasks
whose accuracy hardly beat random guesses.
We also identify a simple metrics
to predict whether a given task belongs
the high-quality category or the low-quality
category. We hope our study shed lights
on future design of signatures for malware
detection.

%Malware detectors mainly rely on signatures manually extracted by security researchers. 
%ssdeep only takes static binary executable as inputs. 
%If we can build malware detector based on ssdeep similarity, 
%We can reduce or even eliminate manual efforts in malware detection. 


\subsection{Data Collection}

\begin{table}
  \centering
  \scriptsize
  \begin{tabular}{clcc}
    \toprule
{\bf Index} & {\bf Microsoft Tag} & {\bf \# of Malwares} & {\bf \% of Tailing} \\
\midrule                                                                                                                                                                                                                                           
1  &  SoftwareBundler:Win32/Penzievs 	& 49380	   & 3.51\%  \\
2  &  Adware:Win32/Hotbar               & 132161   & 1.88\%  \\
3  &  TrojanDropper:Win32/Lamechi!rfn	& 39205	   & 0.01\%  \\
4  &  Virus:Win32/Nabucur.D	            & 1190132  & 99.95\% \\
5  &  Virus:Win32/Virut.BO	            & 84600	   & 21.84\% \\
6  &  Worm:Win32/Mydoom.L@mm	        & 76259	   & 0\%     \\
7  &  Virus:Win32/Ramnit.I              & 412052   & 3.07\%  \\
8  &  Trojan:Win32/Dorv.A               & 54324    & 2.80\%  \\
9  &  Trojan:Win32/Dynamer!ac           & 145402   & 65.17\% \\
10 &  Virus:Win32/Ramnit.A              & 181524   & 26.79\% \\   

\bottomrule
   \end{tabular}
  %\nocaptionrule
  \caption{Malware Family Information.
  \footnotesize{(Malware families used in our clustering and classification experiments. \# of Malwares: \# of distinct malwares we collected for each malware family from 05/07/2016 to 09/06/2016. \% of Tailing: \% of tailing examples after we sampled 10000 malwares for each malware family from our collection.  
  ) 
}
 }
  \label{tab:benchmarks}
\end{table}


\begin{table*}
\centering
\footnotesize
\begin{tabular}{cccccccccccc}
 \toprule
  & \multicolumn{9}{c}{Clustering} &\multicolumn{2}{c}{Classification}\\
\cline{1-1}
\cline{2-10}
\cline{11-12}
 \bf{Index}             & {\bf 0.1}  & {\bf 0.2} & {\bf 0.3} & {\bf 0.4} & {\bf 0.5}  & {\bf 0.6} & {\bf 0.7} & {\bf 0.8} & {\bf 0.9} & {\bf Best k} & {\bf Precision} \\
              \midrule  
          1   & 2483       & 575       &  503      &   456     & 443        & 381       & 356       & 356       & 356       &     1        & 81.27\%  \\
          2   & 4192       & 762       &  635      &   588     & 580        & 534       & 526       & 526       & 526       &     1        & 82\%     \\
          3   &  5         & 4         &  3        &    2      & 2          & 2         & 2         & 2         & 2         &     1        & 82.55\%  \\
          4   & 10000      & 9999      & 9998      & 9997      & 9997       & 9997      & 9997      & 9997      & 9997      &     2        & 59.81\%  \\
          5   & 8688       & 7727      & 6630      & 5462      & 4699       & 3690      & 3103      & 3028      & 3028      &     1        & 76.23\%   \\
          6   & 5807       & 3915      & 1590      & 373       & 18         & 1         & 1         & 1         & 1         &     3        & 81.79\%   \\
          7   & 5429       & 3651      & 2703      & 1615      & 726        & 400       & 369       & 362       & 362       &     1        & 81.79\%   \\
          8   & 2721       & 1400      & 871       & 642       & 549        & 420       & 399       & 396       & 396       &     1        & 82.26\%    \\
          9   & 8500       & 8199      & 8012      & 7808      & 7597       & 7327      & 7171      & 7150      & 7150      &     1        & 64.41\%    \\
          10  & 8075       & 7241      & 6343      & 5193      & 4382       & 3800      & 3506      & 3480      & 3480      &     1        & 74.06\%    \\

\bottomrule
\end{tabular}
\caption{Clustering and Classification Results. \footnotesize{(Numbers of resulting clusters under distance threshold from 0.1 to 0.9 are shown in \bf{Clustering} column.
K with best precision during cross validation and precision of knn with best k are shown in \bf{Classification} column.)}
}
\end{table*}

Our study uses features based on ssdeep~\cite{ssdeep}, a program to compute fuzzy hashes. 
Similarity between calculated hash strings can serve as an estimation for similarity between the two original files. 
ssdeep hash strings are also provided for each submitted file with other metadata fields by \vt. 

We focus on classifying each malware into
different malware families. Because Microsoft 
has a good reputation in detecting PE malwares~\cite{SongAPsys2016}, we create training
data using its assignment. For each detected malware, Microsoft assigns it a tag, which contains type, platform, family, and variant information~\cite{microsoft}. 
We divide PE malwares detected by Microsoft engine into different groups, and malwares in the same group share the same Microsoft malware tag. 
We sample 10 groups, each of which with more than 10000 malwares.  
Microsoft tag and number of malwares in each sampled group we collected from \vt{} are shown in Table~\ref{tab:benchmark}. 
For each group, we sample 10000 malwares, and use these malwares in our following experiments. 


\subsection{Classification Accuracy}

We build an classifier as follows. \ce{XXXX LINHAI, ADD IN THE PROTOCOL.}

Table~\ref{tab:results} shows the classification
result and Figure~\ref{fig:moredata} shows
the relationship between the amount of
training data and the accuracy of classifiers. 
We see that for five out of ten 
malware families, we achieve an accuracy
higher than 80\%. Moreover, as indicated
by Figure~\ref{fig:moredata}, when more
training data are available for these
families, we expect the accuracy to be even
higher. On the other hand, for
families such as ``Virus:Win32/Nabucur.D'',
the classification accuracy are significant
lower. We get an accuracy of 59\% when
\ce{the accuracy of random guesses would be 50\%. LINHAI IS THIS RIGHT?}
\ce{LINHAI, ADD ONE SENTENCE ABOUT THE REASON.}

\paragraph*{Discussion: Tailing Malwares}

We identify one metrics to predict whether
a classification task falls into the
high-accuracy category or low-accuracy
category. The intuition is that the probability
that a given sample has similar samples in
the training set is a proxy of the upper bound
of accuracy that we can expect. Therefore,
we compute the percentage of tailing malwares in each group--We call malwares, which have 0 similarity with all the other samples in the same group, as tailing malwares. 
The percentage of tailing malwares for each sampled group is also shown in Table~\ref{tab:benchmark}. 

\ce{LINHAI, ADD IN THAT \#TRAILING-VS.ACCURACY FIGURE AND ARGUE ABOUT CORRELATION HERE.}

\paragraph*{Discussion: Challenge of Applying Nystrom Methods for Kernel Machines}

\begin{table*}
\centering
\footnotesize
{
\begin{tabular}{cccccccccccc}
 \toprule
  & \multicolumn{9}{c}{Clustering} &\multicolumn{2}{c}{Classification}\\
\cline{1-1}
\cline{2-10}
\cline{11-12}
 \bf{Index}             & {\bf 0.1}  & {\bf 0.2} & {\bf 0.3} & {\bf 0.4} & {\bf 0.5}  & {\bf 0.6} & {\bf 0.7} & {\bf 0.8} & {\bf 0.9} & {\bf Best k} & {\bf Precision} \\
              \midrule  
          1   & 2483       & 575       &  503      &   456     & 443        & 381       & 356       & 356       & 356       &     1        & 81.27\%  \\
          2   & 4192       & 762       &  635      &   588     & 580        & 534       & 526       & 526       & 526       &     1        & 82\%     \\
          3   &  5         & 4         &  3        &    2      & 2          & 2         & 2         & 2         & 2         &     1        & 82.55\%  \\
          4   & 10000      & 9999      & 9998      & 9997      & 9997       & 9997      & 9997      & 9997      & 9997      &     2        & 59.81\%  \\
          5   & 8688       & 7727      & 6630      & 5462      & 4699       & 3690      & 3103      & 3028      & 3028      &     1        & 76.23\%   \\
          6   & 5807       & 3915      & 1590      & 373       & 18         & 1         & 1         & 1         & 1         &     3        & 81.79\%   \\
          7   & 5429       & 3651      & 2703      & 1615      & 726        & 400       & 369       & 362       & 362       &     1        & 81.79\%   \\
          8   & 2721       & 1400      & 871       & 642       & 549        & 420       & 399       & 396       & 396       &     1        & 82.26\%    \\
          9   & 8500       & 8199      & 8012      & 7808      & 7597       & 7327      & 7171      & 7150      & 7150      &     1        & 64.41\%    \\
          10  & 8075       & 7241      & 6343      & 5193      & 4382       & 3800      & 3506      & 3480      & 3480      &     1        & 74.06\%    \\

\bottomrule
\end{tabular}
}
 \mycaption{tab:results}{Clustering and Classification Results.}
{\footnotesize{(Numbers of resulting clusters under distance threshold from 0.1 to 0.9 are shown in \bf{Clustering} column. K with best precision during cross validation and precision of knn with best k are shown in \bf{Classification} column.)}}
%\caption{Clustering and Classification Results. \footnotesize{(Numbers of resulting clusters under distance threshold from 0.1 to 0.9 are shown in \bf{Clustering} column.
%K with best precision during cross validation and precision of knn with best k are shown in \bf{Classification} column.)}
%}
\end{table*}

Our previous experiment uses k-means classifier.
In this section, we discuss the potential of using
more sophisticated classifiers such as
support vector machines. One challenge
of applying kernel machines is to
approximate the kernel matrix whose
size is quadratic to the number of samples
in the training set. Nystrom Methods~\cite{clustering-purpose} are popular ways to
approximate a kernel matrix with clusters.

To understand the potential of applying Nystrom Methods, we run hierarchical clustering~\cite{hcluster} on our data.
Hierarchical clustering starts with each instance as a cluster, 
and then it iteratively merge two clusters with minimum distance 
until distance threshold or cluster number threshold is reached. 
We use distance as threshold. 
Given two malwares, 
we use 1 to minus their ssdeep similarity to calculate distance between the two malwares. 
We calculate single linkage distance as distance between two clusters. 
Single linkage
distance~\cite{single-link} when we need to compute distance between two clusters. 

We change distance threshold from 0.1 to 0.9, 
and count resulting clusters under each experiment. 
Experimental results are shown in Table~\ref{tab:results}. 
As we increase distance threshold, the number of resulting clusters decreases for each sampled group. 
The number of resulting clusters is always bounded by the number of tailing in each group. 
If we want to change a ssdeep hash string to a feature vector, 
based on distance of the string to the center of all clusters in training set, 
the size of the resulting feature vector would have a very large variance across different malware groups. 
\ce{This illustrates a challenge of directly applying
classic Nystrom method, however, we believe
it is possible to develop new approaches to
accommodate this observation in the future.}

\begin{figure*}
\centering
\subfloat[]{\includegraphics[width=0.16\linewidth]{figure/svm/0}\label{fig:moredata1}} 
\subfloat[]{\includegraphics[width=0.16\linewidth]{figure/svm/1}\label{fig:moredata2}}
\subfloat[]{\includegraphics[width=0.16\linewidth]{figure/svm/2}\label{fig:moredata3}} 
\subfloat[]{\includegraphics[width=0.16\linewidth]{figure/svm/3}\label{fig:moredata4}}
\subfloat[]{\includegraphics[width=0.16\linewidth]{figure/svm/4}\label{fig:moredata5}} \\ 

\subfloat[]{\includegraphics[width=0.16\linewidth]{figure/svm/5}\label{fig:moredata6}} 
\subfloat[]{\includegraphics[width=0.16\linewidth]{figure/svm/6}\label{fig:moredata7}}
\subfloat[]{\includegraphics[width=0.16\linewidth]{figure/svm/7}\label{fig:moredata8}} 
\subfloat[]{\includegraphics[width=0.16\linewidth]{figure/svm/8}\label{fig:moredata9}}
\subfloat[]{\includegraphics[width=0.16\linewidth]{figure/svm/9}\label{fig:moredata10}}
\subfloat[]{\includegraphics[width=0.16\linewidth]{figure/svm/10}\label{fig:moredata11}}
\caption{Potential for 0.5 V bias.} 
\label{fig:EcUND} 
\end{figure*} 

%\subsection{Classification Experiments}


\subsection{Discussion}

\ce{ LINHAI 
1. Future work about indexing ssdeep string
2. Upper bound
3. More data better results
4. Need human effort or dynamic information 
5. security of using ssdeep
}


\subsection{Clustering Experiments}


Before conducting classification experiments, 
we run clustering on each group firstly. 
There are two purposes for clustering experiments.
First, we want to understand whether malwares in each group looks similar to each other, 
and understand possible upper bound of precision in malware classification.
Second, some research findings~\cite{clustering-purpose} show that changing instances to feature vectors, 
based on each instance’s distance to the center of each cluster, 
can improve the speed of classification.  

The clustering algorithm we use is hierarchical clustering~\cite{hcluster}.
Hierarchical clustering starts with each instance as a cluster, 
and then it iteratively merge two clusters with minimum distance 
until distance threshold or cluster number threshold is reached. 
We use distance as threshold. 
Given two malwares, 
we use 1 to minus their ssdeep similarity to calculate distance between the two malwares. 
We calculate single linkage distance as distance between two clusters. 
Single linkage
distance~\cite{single-link} when we need to compute distance between two clusters. 

We change distance threshold from 0.1 to 0.9, 
and count resulting clusters under each experiment. 
Experimental results are shown in Table~\ref{tab:results}. 
As we increase distance threshold, the number of resulting clusters decreases for each sampled group. 
The number of resulting clusters is always bounded by the number of tailing in each group. 
If we want to change a ssdeep hash string to a feature vector, 
based on distance of the string to the center of all clusters in training set, 
the size of the resulting feature vector would have a very large variance across different malware groups. 
This is the reason why we do not apply method proposed by~\citet{clustering-purpose} in our classification experiments. 


\subsection{Classification Experiments}

We use knn~\cite{knn} as our classification algorithm.
We design several experiments to understand performance of 
the combination of knn and ssdeep distance (or similarity) when handling different classification tasks. 

We first run two-class classification for each group. 
We use the 10000 sampled malwares from each group as positive examples. 
We sample 10000 benign files as negative examples. 
Benign files are files submitted to \vt{}, but are not labeled as malwares by any anti-virus engines. 
We randomly choose 5000 malwares and 5000 benign files as training examples, 
and use the left malwares and benign files as testing set. 
We run cross validation for k from 1 to 10 to pick up best k, 
and use the best k to test knn with the testing set. 

Experimental results from two-class 
classification for each group are shown in Table~\ref{tab:results}.
Except group 4 and group 6, 1 is the best k for all other groups. 
During cross validation, precision roughly decreases, with k increasing. 
This means that given a testing instance, 
considering more instances in training set usually bring more noise. 
The precision on testing sets with best k varies from 59.81\% to 82.55\%.

We then run ten-class classification.
We randomly choose 100 malwares from each group, 
and put 50 malwares in training set and 50 malwares in testing set.
We also run cross validation to pick up best k from 1 to 10. 
Similar to two-class classification, precision roughly decreases, with k increasing, 
and the best k is 1. 
The precision on the testing set with $k = 1$ is 70.15\%, 
which is roughly equal to average percentage of tailing in all sampled malware groups. 

For each experiment, we investigate 
how the size of training set influences precision.
We keep test sets unchanged.  
We increase the size of training set from 50 to 10000, 
while keeping the number of examples from different classes equal to each other.  
We use the best k got from earlier cross validation.
How precision changes with the size of training set is shown in Figure~\ref{fig:moredata}. 
Precision increases for all experiments, as we put more data in training set.   

We also design several other two-class classification tasks. 
We classify malwares from Group 7 with malwares from Group 10. 
They are malwares from the same class Ramnit, but in different variant. 
The precision is 85.31\%. 
We classify malwares from Group 3 with malwares from Group 7 and Group 10, 
and classify malwares from Group 5 with malwares from Group 7 and Group 10. 
These two classifications try to classify malwares in different families. 
We get precision 94.15\% and 85.82\%, respectively. 
We classify malwares from Group 8 and Group 9 with malwares from Group 4, Group 5, Group 7 and Group 10, 
because we want to classify malwares in different types, one is Trojan and the other is Virus. We get precision 77.04\%. 
All experimental setting are the same as what we did during classifying malwares in each group with benign files. 

\subsection{Discussion}
In our experiments, 1 is almost always the best k value. 
So using ssdeep similarity to conduct malware classification 
is roughly to search the most similar example in the training set for each testing example.
In our knn experiment, we compare the testing example with every example in training set. 
In the future, indexing techniques could be leveraged to reduce the complexity of identify 
the most similar example in training set from $O(n)$ to $O(1)$, where n is the number of examples in training set. 

Our experimental results show that with more data in training set, 
we can get better precision, but the precision is upper bounded by the percentage of tailing in test set. 