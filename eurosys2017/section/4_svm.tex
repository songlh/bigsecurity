\section{Malware Classification}
\label{sec:ssdeep}

%Machine learning systems are just as good
%as the data they consumed. 
One potential use case of the massive amount of data 
available on \vt{} is to build 
machine learning models for applications
such as malware classifications and
clustering. In this section, we study
the question that {\em ``Is the data
provided by \vt{} enough to support
classifications and clusterings tasks?''}

Our study reveals both positive and
negative answers to this question.
For some classification tasks, we are able
to build an automatic classifier whose
accuracy is higher than 80\% by just
using the static fuzzy hash strings provided by \vt{}
and we expect the quality to keep increasing
given more data; on the other hand,
we identify some classification tasks
whose accuracy hardly beat random guesses.
We also identify a simple metrics
to predict whether a given task belongs
the high-quality category or the low-quality
category. We hope our study shed lights
on future design of signatures for malware
detection.

%Malware detectors mainly rely on signatures manually extracted by security researchers. 
%ssdeep only takes static binary executable as inputs. 
%If we can build malware detector based on ssdeep similarity, 
%We can reduce or even eliminate manual efforts in malware detection. 


\subsection{Data Collection}

\begin{table}
  \centering
  \scriptsize
  \begin{tabular}{clcc}
    \toprule
{\bf Index} & {\bf Microsoft Tag} & {\bf \# of Malwares} & {\bf \% of Tailing} \\
\midrule                                                                                                                                                                                                                                           
1  &  SoftwareBundler:Win32/Penzievs 	& 49380	   & 3.51\%  \\
2  &  Adware:Win32/Hotbar               & 132161   & 1.88\%  \\
3  &  TrojanDropper:Win32/Lamechi!rfn	& 39205	   & 0.01\%  \\
4  &  Virus:Win32/Nabucur.D	            & 1190132  & 99.95\% \\
5  &  Virus:Win32/Virut.BO	            & 84600	   & 21.84\% \\
6  &  Worm:Win32/Mydoom.L@mm	        & 76259	   & 0\%     \\
7  &  Virus:Win32/Ramnit.I              & 412052   & 3.07\%  \\
8  &  Trojan:Win32/Dorv.A               & 54324    & 2.80\%  \\
9  &  Trojan:Win32/Dynamer!ac           & 145402   & 65.17\% \\
10 &  Virus:Win32/Ramnit.A              & 181524   & 26.79\% \\   

\bottomrule
   \end{tabular}
  %\nocaptionrule
  \caption{Malware Family Information.
  \footnotesize{(Malware families used in our clustering and classification experiments. \# of Malwares: \# of distinct malwares we collected for each malware family from 05/07/2016 to 09/06/2016. \% of Tailing: \% of tailing examples after we sampled 10000 malwares for each malware family from our collection.  
  ) 
}
 }
  \label{tab:benchmarks}
\end{table}


\begin{table*}
\centering
\footnotesize
\begin{tabular}{cccccccccccc}
 \toprule
  & \multicolumn{9}{c}{Clustering} &\multicolumn{2}{c}{Classification}\\
\cline{1-1}
\cline{2-10}
\cline{11-12}
 \bf{Index}             & {\bf 0.1}  & {\bf 0.2} & {\bf 0.3} & {\bf 0.4} & {\bf 0.5}  & {\bf 0.6} & {\bf 0.7} & {\bf 0.8} & {\bf 0.9} & {\bf Best k} & {\bf Precision} \\
              \midrule  
          1   & 2483       & 575       &  503      &   456     & 443        & 381       & 356       & 356       & 356       &     1        & 81.27\%  \\
          2   & 4192       & 762       &  635      &   588     & 580        & 534       & 526       & 526       & 526       &     1        & 82\%     \\
          3   &  5         & 4         &  3        &    2      & 2          & 2         & 2         & 2         & 2         &     1        & 82.55\%  \\
          4   & 10000      & 9999      & 9998      & 9997      & 9997       & 9997      & 9997      & 9997      & 9997      &     2        & 59.81\%  \\
          5   & 8688       & 7727      & 6630      & 5462      & 4699       & 3690      & 3103      & 3028      & 3028      &     1        & 76.23\%   \\
          6   & 5807       & 3915      & 1590      & 373       & 18         & 1         & 1         & 1         & 1         &     3        & 81.79\%   \\
          7   & 5429       & 3651      & 2703      & 1615      & 726        & 400       & 369       & 362       & 362       &     1        & 81.79\%   \\
          8   & 2721       & 1400      & 871       & 642       & 549        & 420       & 399       & 396       & 396       &     1        & 82.26\%    \\
          9   & 8500       & 8199      & 8012      & 7808      & 7597       & 7327      & 7171      & 7150      & 7150      &     1        & 64.41\%    \\
          10  & 8075       & 7241      & 6343      & 5193      & 4382       & 3800      & 3506      & 3480      & 3480      &     1        & 74.06\%    \\

\bottomrule
\end{tabular}
\caption{Clustering and Classification Results. \footnotesize{(Numbers of resulting clusters under distance threshold from 0.1 to 0.9 are shown in \bf{Clustering} column.
K with best precision during cross validation and precision of knn with best k are shown in \bf{Classification} column.)}
}
\end{table*}

Our study uses features based on ssdeep~\cite{ssdeep}, a program to compute fuzzy hashes. 
Specifically, ssdeep \yiying{fill here, what ssdeep exactly is and does}
Similarity between calculated hash strings can serve as an estimation for similarity between the two original files. 
\yiying{why can it estimate? we need some deeper understanding of the hash or need to be mroe careful when saying things like this. people's assumption is that hash cannot represent original data and is just a highly compressed string.}
\vt\ provides ssdeep hash strings as one of its metadata files and we collected them together with other metadata.

%We focus on classifying each malware into
%different malware families. 
We create training and testing
datasets using the detection results from Microsoft,
since Microsoft is more reliable in successfully detecting PE malwares~\cite{SongAPsys2016} than other anti-virus engines.
For each detected malware, 
Microsoft assigns a tag that contains type, platform, family, 
and variant information~\cite{microsoft}. 

We divide PE malwares detected by Microsoft engine into different groups, 
and malwares in the same group share the same Microsoft malware tag. 
We randomly sample 10 groups that have more than 10000 malwares.  
Table~\ref{tab:benchmark} presents the Microsoft tag sand the number of malwares in each sampled group we collected from \vt{}. 
Within each group, we randomly sample 10000 malwares and use these malwares in our classification and clustering experiments. 

\subsection{Classification Methodology and Results}
We use an API that \vt\ provides to compute the similarity between ssdeep string. \yiying{is the previous sentence correct?}
%ssdeep's compare API takes two ssdeep hash strings as inputs, 
%and return similarity between these two strings. 
We use 1 minus ssdeep similarity to get ssdeep distance.  

\subsubsection{KNN Classification}

We first use KNN~\cite{knn}, a simple classifier, to classify XXX.
KNN does ....
\yiying{full name of KNN, a simple description of it}
We design several experiments to understand the accuracy of 
the combination of KNN and ssdeep 
distance (or similarity) when handling different classification tasks. 

\noindent{\underline{\textit{Pair-wise two-class classification.}}}
We first build a two-class classifier for each of the 10 sampled groups. 
We use the 10000 sampled malwares in each group as positive examples (positive means being labeled as malware)
and randomly sample 10000 benign files as negative examples. 
Benign files are files submitted to \vt{} but are not labeled as malwares by any anti-virus engines. 
We randomly choose 5000 malwares and 5000 benign files for training and use the remaining files for testing.

We run cross validation for $k$ from 1 to 10 to pick up best $k$ value
and use the best $k$ value to test KNN using the testing set. 
Table~\ref{tab:benchmark} presents the best $k$ used for testing and classification results.
$k=1$ is always the best $k$ value for testing, except for two groups.

The classification accuracy of the two-class classifier differs in different groups.
For five out of ten malware groups, we achieve an accuracy higher than 80\%.
On the other hand, for
groups such as ``Virus:Win32/Nabucur.D'',
the classification accuracy is significantly lower. 
We get an accuracy of 59\% when
the accuracy of random guesses would be 50\%.
Overall, for the two-class classifier, 
there are not enough malwares and benign files in the training set to generate high-accuracy results. 

\noindent{\underline{\textit{Compelete ten-class classification.}}}
We also build a ten-class classifier.
We randomly choose 2000 malwares from each group and put half of them in the training set and the remaining half in the testing set. 
We run cross validation to pick up best $k$ from 1 to 10. 
The accuracy on the testing set with the best $k$ value 1 is 70.2\%. 

\begin{figure*}
\centering
\subfloat[]{\includegraphics[width=0.16\linewidth]{figure/svm/0}\label{fig:moredata1}} 
\subfloat[]{\includegraphics[width=0.16\linewidth]{figure/svm/1}\label{fig:moredata2}}
\subfloat[]{\includegraphics[width=0.16\linewidth]{figure/svm/2}\label{fig:moredata3}} 
\subfloat[]{\includegraphics[width=0.16\linewidth]{figure/svm/3}\label{fig:moredata4}}
\subfloat[]{\includegraphics[width=0.16\linewidth]{figure/svm/4}\label{fig:moredata5}} \\ 

\subfloat[]{\includegraphics[width=0.16\linewidth]{figure/svm/5}\label{fig:moredata6}} 
\subfloat[]{\includegraphics[width=0.16\linewidth]{figure/svm/6}\label{fig:moredata7}}
\subfloat[]{\includegraphics[width=0.16\linewidth]{figure/svm/7}\label{fig:moredata8}} 
\subfloat[]{\includegraphics[width=0.16\linewidth]{figure/svm/8}\label{fig:moredata9}}
\subfloat[]{\includegraphics[width=0.16\linewidth]{figure/svm/9}\label{fig:moredata10}}
\subfloat[]{\includegraphics[width=0.16\linewidth]{figure/svm/10}\label{fig:moredata11}}
\caption{Potential for 0.5 V bias.} 
\label{fig:EcUND} 
\end{figure*} 



\begin{figure}[t!]
\begin{center}
\includegraphics[width=2.5in]{figure/accuracy}
\mycaption{fig:accuracy}{The relation between classifer's accuracy and the percentage of tailing.}
{\footnotesize{(How classifier's accuracy changes with the percentage of tailing.)}}
\end{center}
%\vspace{-0.25in}
\end{figure}




\noindent{\underline{\textit{Effect of training set size.}}}
For each designed classifier, we further investigate 
how the size of training sets influences its accuracy.
We keep the test sets unchanged and increase the size of training sets from 50 to 10000.
%The total number of data points in all groups is the same.
We use the best $k$ value got from cross validation in earlier experiments.
Figure~\ref{fig:moredata} presents how accuracy changes with the size of training sets. 
As expected, for all classifiers, accuracy increases as there are more data in the training sets.
This constant increase in accuracy implies that with more data from \vt,
the classification accuracy is expected to be even higher.

\noindent{\underline{\textit{Classifications with semantics labels.}}}
To make classification based on more criteria, 
we further design a set of two-class classifications. 

First, we test the classfication of malwares with the same {\em class} but with different {\em variants} in the class.
Specifically, we classify malwares from Group 7 with malwares from Group 10. 
They are both from the same class ``Ramnit'', but in different variants. 
The accuracy we gotis 85.3\%. 

We then target to classify malwares in different {\em families}.
To do this, we classify malwares from Group 3 with malwares from Group 7 and Group 10
and classify malwares from Group 5 with malwares from Group 7 and Group 10. 
We get accuracies of 94.2\% and 85.8\%, respectively. 

Finally, we test the classification of different {\em types} of malware.
Specifically, we classify malwares from Group 8 and Group 9 with malwares from Group 4, Group 5, Group 7, and Group 10, 
the former groups have Trojan malwares and the latter group contains Virus. 
For this classification, we get 77.0\% accuracy. 
%We use the same experimental setting as previous two-class classifiers.  

{\bf Observation 9:}
{\em Different types of classification on \vt\ file ssdeep hash strings result in different accuracy, and increasing the training set size results in higher accuracy.}

\subsection{Tailing Malwares}
From the above study of various types of classification,
we find that using ssdeep hash strings to learn the similarities of \vt\ submission files 
is not always accurate.
But in some cases, the accuracy is very high.
To make use of these high-accuracy classification,
we need a good technique to seperate them from low-accuracy classification.

We identify one metric to predict whether
a classification task falls into the
high-accuracy category or the low-accuracy
category. The intuition is that the probability
that a given sample has similar samples in
the training set is a proxy of the upper bound
of accuracy that we can expect. Therefore,
we compute the percentage of tailing malwares in each group.
We call malwares that have 0 similarity with all the other samples in the same group {\em tailing malwares}. 
Table~\ref{tab:benchmark} lists the percentage of tailing malwares for each sampled group. 
Figure~\ref{fig:accuracy} plots the classification accuracy against the percentage of tailing malwares.

{\bf Observation 10:} 
{\em The percentage of tailing malwares
can predict whether a given malware classification task is likely to have high or low accuracy.}

\begin{figure}[t!]
\begin{center}
\includegraphics[width=2.5in]{figure/cluster}
\mycaption{fig:cluster}{The relation between resulting clusters and distance threshold.}
{\footnotesize{(How the number of resulting clusters change with different distance threshold.)}}
\end{center}
%\vspace{-0.25in}
\end{figure}


\subsection{SVM Classification}
%\noindent{\underline{\textit{SVM classification.}}}
Our previous experiments use KNN.
We now discuss the potential of using
more sophisticated classifiers such as
support vector machines (SVM). 
SVM could take distance matrix as input.
However, the problem is that the size of distance matrix is quadratic to the number of samples in the training set. 
Nystrom methods~\cite{clustering-purpose} are popular ways to
approximate a distance matrix with clusters.
By using Nystrom methods, each instance is transformed to a feature vector, where each dimension is the distance to a cluster center. 

To understand the potential of applying Nystrom methods, we run hierarchical clustering~\cite{hcluster} on our data.
Hierarchical clustering starts with each instance as a cluster, 
and then it iteratively merges two clusters with minimum distance 
until distance threshold or cluster number threshold is reached. 
We use distance as threshold. 
We use single linkage distance~\cite{single-link} as distance between two clusters. 

We change distance threshold from 0.1 to 0.9, 
and count resulting clusters under each experiment. 
Experimental results are shown in Figure~\ref{fig:cluster}. 
As we increase distance threshold, the number of resulting clusters decreases for each group. 
If we want to change a ssdeep hash string to a feature vector, 
based on distance of the string to the center of all clusters in training set, 
the size of the resulting feature vector would have a very large variance across different malware groups. 
This illustrates a challenge of directly applying
classic Nystrom method, however, we believe
it is possible to develop new approaches to
accommodate this observation.


\subsection{Discussion}

In our experiments, 1 is almost always the best k value. 
So using ssdeep similarity to conduct malware classification 
is roughly to search the most similar example in the training set for each testing example.
In our knn experiment, we compare a testing example with every example in training set. 
In the future, indexing techniques could be leveraged to reduce the complexity of identify 
the most similar example in training set from $O(n)$ to $O(1)$, where n is the number of examples in training set. 

Our experimental results show that with more data in training set, 
we can get better precision. 
For a testing instance, if there are similar instances in training set, 
classifier based on ssdeep similarity can precisely classify the instance. 
\vt{} contains huge amount of submitted files. We suggest \vt{} provides an extra API, 
where users only need to submit ssdeep strings for suspicious files, 
and this could improve privacy protection for \vt{}’s users.  



