In this paper, we conducted a thorough empirical study on real-world malwares and anti-virus 
engines by examining the largest real malware repository, \vt. 
Our study is performed in three dimensions.
First, we study the correlation between malwares’ metadata and their detection rates. 
Second, we model how influence propagates across different anti-virus vendors. 
Finally, we explore the feasibility to build machine learning malware detectors based only on hash values of files.
Our study results confirmed a set of hypothesis and anecdotal assumptions
and revealed a few surprising new insights into malwares and anti-virus engines.
Together with our machine learning framework, these results can shed light on future research directions
and assist both anti-virus vendors and normal users in their fight against malwares.
