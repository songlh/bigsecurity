\section{Related Works}
\label{sec:related}

\subsection{Leveraging Source Code Repositories}

Recently, there are many works on exploring how to leverage ``big code'' repositories, 
such as GitHub, BitBucket, and CodePlex. 
SLANG~\cite{code-completion} trains statistical models by using extracted sequences of API calls from large code bases, 
and applies trained models to fill uncompleted programs with call innovations. 
JSNICE~\cite{big-predicting} aims to predict identifier types and obfuscated identifier names for JavaScript programs. 
A score function based on learned CRF model is optimized to make all predictions. 
Phrase-based statistical translation approaches are used by \citet{big-translation} to translate C\# programs to Java programs. 

As a summary, all these techniques are utilizing source code repositories, 
and their techniques are built based on analyzing source codes. 
However, what we are leveraging is a malware repository, 
and we rely on malwares’ metadata information and ssdeep hash strings calculated from binary executables.


\subsection{Analyzing and Leveraging VirusTotal}

Research community starts to pay more attention to VirusTotal repository. 

~\citet{SongAPsys2016} download one-month data from VirusTotal, 
and study PE malwares’ basic properties, 
temporal properties, and distribution properties by using their collected data.   
Similar to our work, they also focus on PE files, and study some general properties. 
Different from our work, they simply rely on detection results from Microsoft anti-virus engines.
We study correlations between metadata and detection results from all engines, 
and we also evaluate different detection engines by modeling influence among different engines.  

Researchers~\cite{huangvt2016bigdata, neeles} notice that some malware writers use VirusTotal as testing platform.
They explore this phenomenon, and build techniques to identify malware writers. 
~\citet{huangvt2016bigdata} download 4 month VirusTotal metadata to identify Android malware development cases. 
Their technique would alert suspicious source ids firstly, 
and they conduct program analysis to further validate whether 
these source ids are really testing their Android malware prototypes on VirusTotal. 
~\citet{neeles} try to identify windows malware development cases on Anubis. 
They also use data on VirusTotal to validate their findings. 
Different from our study, their works focus on submission history of each source id.  

~\citet{betterGT} applies various supervised and unsupervised 
technique to aggregate various labels from different anti-virus engines into one ground truth label 
on a dataset collected from VirusTotal. 
Their goal is to find better ground truth label for distinct samples, 
while our work tries to model influence among different anti-virus vendors.  

%3. Code classification and clustering

