\section{Methodology}
\label{sec:meth}




\begin{table}[h!]
\centering
\footnotesize
{
%\begin{tabular}{@{\hspace{3pt}}l@{\hspace{3pt}}|@{\hspace{3pt}}c@{\hspace{3pt}}}
\begin{tabular}{l|l}
\hline
Metadata Fields & Explanation \\
\hline                            
%\cline{1-1}
{\bf name}      & file name of the submitted sample \\
{\bf timestamp} & timestamp for when the submission was made \\
{\bf source\_country} & the country from which the submission was made \\
{\bf source\_id} & user id of whom made the submission\\
{\bf tags} & VirusTotal tag \\
{\bf link} & where to download the submitted sample \\
{\bf size} & file size \\
{\bf type} & file type \\
{\bf first\_seen} & when the same sample was first submitted \\
{\bf last\_seen} & when the same sample was last submitted \\
{\bf hashes} & including sha1, sha256, vhash, md5, and ssdeep\\
{\bf total} & how many engines analyze the sample\\
{\bf positives} & how many engines identify the sample as malicious \\
{\bf positives\_delta} & changes in {\bf positives} fields \\
{\bf report} & detailed detection report from each AV engine \\
%\multicolumn{2}{|l|}
\hline

\end{tabular}
}
\caption{Metadata fields for each submission retrieved from the VirusTotal private API.}
\label{tab:fields}
\end{table}



We download submission reports' metadata from the private API of VirusTotal. 
Table~\ref{tab:fields} shows all metadata fields and their meaning. 
In this paper, we only focus on Portable Executable (PE) files, 
and we leave the analysis of other types of malicious files for the future. 
We filter all records by the tag field. If the tag field contains either ``peexe'' or ``pedll'' tag, the record is considered to be a PE file. 
Then we rely on Microsoft antivirus engine to judge whether the PE file is a malware and if so its malware family.

It is possible that the VirusTotal private API returns redundant reports, 
and we use the combination of md5 and timestamp
 to detect and merge redundant reports.

We downloaded all reports in November 2015, collecting a total of 43 million reports of which 4.7 million are PE malwares. 
The number of reports and PE malwares submitted each day are shown in Figure~\ref{fig:subnum}.

After removing redundant submissions, we find that most malwares were submitted only once to VirusTotal. 
4 million out of the total 4.7 million PE malware submissions are distinct. 
On average, each PE malware was submitted 1.17 times to VirusTotal. We believe that most malwares are encountered by more than one VirusTotal user. 
One possible reason for this observation is that VirusTotal users 
tend to check whether their samples have already been submitted
before making their own submissions.

\textit{\underline{Threats to Validity.}}
Similar to all previous empirical studies, all our findings, experimental results, 
and conclusions need to be considered with our methodology in mind. 

The VirusTotal private API only tracks which submission reports are sent to each downloader approximately, 
and there is no guarantee that all submission reports on VirusTotal can be downloaded successfully. 
It could be possible that we missed some malwares submitted to VirusTotal. 
Also, we simply leverage Microsoft antivirus engine to decide whether a submission is malicious or not, 
and it is possible that Microsoft antivirus engine cannot make this decision precisely. 
However, how to get a precise label for a PE file is out of the scope of this paper.  
Although there is a huge amount of malwares on VirusTotal, we believe there are malwares never submitted to VirusTotal, 
and there are malwares submitted much later than when they appear in the real world.
However, there are no conceivable ways to study these malwares. 
We believe that the malwares in our study provide a representative malware sample of the real world. 


