%Title: 
%1. Learning from VirusTotal
%2. A Glimpse of VirusTotal

\section{Introduction}

Malwares grow exponentially~\cite{avtest}. 
AVTest~\cite{avtest} reports that more than 140 million new malwares appeared in 2015. 
Such malwares are posing increasing threats to the human society every day. 
For example, there were almost 2 million attempts to 
steal money from online bank accounts 
with malwares that exploit vulnerabilities in the Adobe Flash player~\cite{kaspersky}. 

Researchers and practitioners continue to build security tools to defend against new malwares,
To assist the design of these tools, it is essential to understand malwares in the real world. 

Previous works on analyzing the behaviors and evolutions of malwares~\cite{ZhouSP2012,GuptaComsnets2009} 
have provided insights 
into how malwares circumvent the detection from existing antivirus techniques and how malware writers create new malwares. 
However, these works only studied a limited amount of malwares that are targeted for certain types of security threats or antivirus engines.
Studying malwares in a large scale and with high diversity can expose new insights beyond these isolated studies.

VirusTotal~\cite{virustotal} is a popular online service that real-world users use to analyze suspicious files and URLs.
It applies more than 50 antivirus engines to each submitted file 
to detect various kinds of malwares including viruses, worms, and trojans. 
It then generates a summary report that includes the detection results of all these engines. 
VirusTotal saves and provides an open access to all user-submitted files and generated reports. 

The VirusTotal repository provides a valuable resource to gain insights into 
the behavior of malware.
%Inspired by previous work on mining software repositories~\cite{GuoICSE2010,bigcode, big-lessons,big-translation,code-completion,big-predicting} 
%in the software engineering and programming languages community, 
%we believe that leveraging data on VirusTotal could also enable many ``big security'' applications.   
%
First, it contains a huge amount of real-world files.
For example, there were more than 40 million suspicious files submitted in November 2015 (Figure~\ref{fig:subnum}). 
Files from VirusTotal were submitted by real-world users from all over the world since 2004
and involve various security threats. 
This amount of diverse data makes VirusTotal a good representitive of malwares in the real world. 

Second, VirusTotal applies a host of state-of-the-art antivirus engines to all submitted files,

%All data on VirusTotal are labeled by state-of-the-art antivirus engines. 

Third, VirusTotal provides rich metadata. 
%VirusTotal updates each antivirus engine every five minutes. 
Besides reporting whether or not a submitted file has virus, 
VirusTotal also captures the exact type of virus, 
There are also online active malware researchers 
who can comment and vote on each submitted file 
and thus serve as an important supplement to antivirus engines. 

There are many research opportunities in mining the VirusTotal repository. 
%List all the possible research questions that one can answer by analyzing VT data.
%For example, ...
For example, we can figure out what vulnerabilities malware writers prefer to exploit, 
and design new systems and programming languages without those vulnerabilities. 
For example, most malwares are created automatically. 
Could we catch malwares' evolution pattern, and automate the updates of antivirus techniques?
For example, there are increasing interests in applying machine learning techniques in different areas. 
Could we leverage VirusTotal data to train a machine learning model and detect malwares we have not seen before? 

Unfortunately, there has been little work in looking at this valuable repository.
In industry, antivirus vendors widely use VirusTotal to identify false negatives 
and false positives in their products. 
However, they only utilize VirusTotal reports separately for each single suspicious file, 
failing to consider correlations among different suspicious files. 
In academia, researchers have begun to pay attention to mining the VirusTotal repository. 
For example, ~\citet{neeles} leverage submission\_id information to identify malware writers 
who use VirusTotal as a test platform. 

We propose to investigate in the VirusTotal repository by ***.
a. studying metadata, including submission information and reports from different vendors, 
to understand general characteristics of malwares in the real world and correlations among different malwares. 
b. downloading malware samples and analyze them to understand their static features and how malwares evolve. 
c. leveraging behavior information on virustotal, and run malwares in virtual machines to understand their dynamic behaviors. 
%I do not quite understand what you want to put here

As a first step, we collected one month of VirusTotal repository data with more than 40 million suspicious files
and conducted an empirical study on the malware dataset on VirusTotal. 
%An empirical study is the prerequisite to conduct data mining on VirusTotal. 
%We first collect more than 40 million suspicious file submissions from VirusTotal.

% this para needs total rewriting
We focus our analysis on Windows executable malwares detected by Microsoft
antivirus engine to guarantee accuracy. 
Our analysis includes general characteristics such as submission frequency and the generation rate of malware families, 
temporal characteristics, and family distribution characteristics. 
We hypothesize that malwares do not appear uniformly across time, rather they appear in bursts. 
To validate our hypothesis, we build a cache-based malware prediction technique that aims to predict malwares in which families would appear in the near future. 
Our malware family cache can achieve a 90\% cache hit rate by only using 100 cache entries.
Our technique would allow antivirus vendors to focus their efforts. 
We also observe that distributions of malware families are highly skewed. 
This observation inspires us to apply a frequent item mining algorithm to 
identify hot malware families. 
We view malware submissions as a stream based on their submission timestamps and 
feed this stream into space saving algorithm~\cite{space-saving}. 
Our solution can precisely answer hot malware family queries in nearly-real time, by using a constant number of counters.  


\begin{figure}[t!]
\begin{center}
\includegraphics[width=2.5in]{figure/nov}
\caption{The number of suspicious files and the number of malwares submitted to VirusTotal in November 2015. }
\label{fig:subnum}
\end{center}
\end{figure}

In summary, this paper makes the following contributions:

\begin{itemize}

\item We are the first to collect data from the VirusTotal repository.
We collected data submitted to VirusTotal in November 2015 
and analyze their general characteristics. 
We find that most malware samples are only submitted once to VirusTotal (Section~\ref{sec:meth}) 
and that roughly 100-400 new malware families appear each day (Section~\ref{sec:temporal}). 


\item We hypothesize that malwares appear in bursts. 
Our cache-based malware prediction technique confirms our hypothesis, 
and it can achieve greater than 90\% prediction precision (Section~\ref{sec:temporal}). 

\item We observe that family distributions of malwares are highly skewed. 
Leveraging this observation, we build a hot malware family mining technique that can identify hot 
malware families by using a constant number of counters (Section~\ref{sec:dist}).

\item We discuss the future research opportunities available with regard to mining data on VirusTotal (Section~\ref{sec:oppo}). 

\end{itemize}


